\documentclass[11pt,a4paper]{article}

% Pakiety
\usepackage[utf8]{inputenc}
\usepackage[T1]{fontenc}
\usepackage[polish]{babel}
\usepackage[left=1.2cm,right=1.2cm,top=1.2cm,bottom=1.2cm]{geometry}
\usepackage{hyperref}
\usepackage{enumitem}
\usepackage{titlesec}
\usepackage{xcolor}
\usepackage{fontawesome5} % Opcjonalnie dla ikon

% Kolory
\definecolor{primary}{RGB}{0, 70, 140} % Profesjonalny niebieski
\definecolor{darktext}{RGB}{40, 40, 40}

% Konfiguracja
\hypersetup{
    colorlinks=true,
    linkcolor=primary,
    urlcolor=primary,
    pdftitle={CV - Jakub Skrzynecki},
    pdfauthor={Jakub Skrzynecki}
}

\setlist{nosep, leftmargin=*}
\pagestyle{empty}
\raggedright

% Formatowanie sekcji
\titleformat{\section}
{\Large\bfseries\color{primary}\uppercase}
{}
{0em}
{}[\titlerule]

\titlespacing*{\section}{0pt}{12pt}{8pt}
\titlespacing*{\subsection}{0pt}{8pt}{4pt}

% Makra
\newcommand{\cvsection}[1]{
    \section{#1}
}

\newcommand{\cventry}[4]{
    \noindent\textbf{#1} \hfill \textbf{#3} \\
    \textit{#2} \hfill \textit{#4} \\
    \vspace{4pt}
}

\newcommand{\cvproject}[3]{
    \noindent\textbf{#1} \hfill \textit{#2} \\
    \textit{#3} \\
    \vspace{4pt}
}

\newcommand{\educationentry}[4]{
    \noindent\textbf{#1} \hfill #2 \\
    \textit{#3} \\
    #4
    \vspace{4pt}
}

\begin{document}

% Nagłówek
\begin{center}
    {\Huge\bfseries\color{darktext} JAKUB SKRZYNECKI} \\[6pt]
    {\Large\color{primary} Junior Cybersecurity Analyst | SOC Trainee | Blue Team} \\[8pt]
    Rzeszów, Polska \\[4pt]
    \href{mailto:sjakubskrzynecki@gmail.com}{sjakubskrzynecki@gmail.com} \quad|\quad +48 662 516 512 \\[4pt]
    \href{https://linkedin.com/in/jakub-skrzynecki-f0r-vv0rk}{LinkedIn: jakub-skrzynecki} \quad|\quad
    \href{https://jakubsx01.github.io/}{Portfolio: jakubsx01.github.io} \quad|\quad
    \href{https://tryhackme.com/p/CyberhelperJS}{TryHackMe: CyberhelperJS}
\end{center}

\vspace{8pt}

% Profil
\cvsection{Profil Zawodowy}
Ambitny student studiów magisterskich na kierunku Cyberbezpieczeństwo z solidnymi podstawami w zakresie obrony sieci, bezpieczeństwa aplikacji webowych i technologii chmurowych. Praktyczne doświadczenie zdobyte poprzez projekty akademickie i symulacje laboratoryjne (TryHackMe, Home Lab). Biegłość w ocenie podatności (OWASP ZAP, OpenVAS), wdrażaniu IAM (Authentik, Azure AD) oraz podstawach reagowania na incydenty. Pasjonat Blue Teaming, dążący do rozwoju w roli analityka SOC lub inżyniera bezpieczeństwa.

% Doświadczenie
\cvsection{Doświadczenie Zawodowe}

\cventry{Praktykant IT/Sieć}{ELTEL NETWORKS ENERGETYKA S.A.}{Lipiec 2023 -- Sierpień 2023}{Widełka, Polska}
\begin{itemize}[leftmargin=1.5em, label=\textbullet]
    \item \textbf{Wsparcie bezpieczeństwa sieci}: Asystowałem przy bezpiecznej konfiguracji infrastruktury sieciowej (routery, switche, AP), zapewniając zgodność z politykami bezpieczeństwa.
    \item \textbf{Zarządzanie zasobami}: Przeprowadziłem inwentaryzację i enumerację zasobów IT, wspierając procesy oceny ryzyka i zarządzania podatnościami.
    \item \textbf{Ciągłość działania}: Weryfikowałem systemy UPS i zasilania awaryjnego, gwarantując niezawodność infrastruktury krytycznej.
    \item \textbf{Wsparcie techniczne}: Rozwiązywałem problemy sprzętowe i programowe użytkowników końcowych, utrzymując wysoką dostępność stacji roboczych.
\end{itemize}

% Projekty
\cvsection{Kluczowe Projekty}

\cvproject{Wdrożenie Identity \& Access Management}{Projekt Akademicki}{Authentik, Microsoft Azure, Docker}
\begin{itemize}[leftmargin=1.5em, label=\textbullet]
    \item \textbf{Wdrożenie SSO}: Skonfigurowałem Authentik jako dostawcę tożsamości (IdP) w Azure, umożliwiając Single Sign-On (SSO) dla wielu aplikacji poprzez \textbf{OIDC} i \textbf{SAML 2.0}.
    \item \textbf{Integracja}: Zintegrowałem usługi z Active Directory (LDAP), centralizując zarządzanie użytkownikami dla symulowanej organizacji (3 grupy, 18 użytkowników).
    \item \textbf{Wzmocnienie bezpieczeństwa}: Zaimplementowałem MFA i ścisłe polityki dostępu, chroniąc przed nieautoryzowanym dostępem.
\end{itemize}

\cvproject{Audyt Bezpieczeństwa Aplikacji Webowej (PlantCare)}{Projekt Akademicki}{OWASP ZAP, Burp Suite, Python}
\begin{itemize}[leftmargin=1.5em, label=\textbullet]
    \item \textbf{Ocena podatności}: Przeprowadziłem kompleksowy audyt bezpieczeństwa w oparciu o OWASP Top 10, identyfikując krytyczne luki, w tym \textbf{Unrestricted File Upload} i XSS.
    \item \textbf{Automatyczne skanowanie}: Wykorzystałem OWASP ZAP do skanowania automatycznego oraz GitHub Dependabot do analizy zależności (SCA).
    \item \textbf{Remediacja}: Opracowałem szczegółowe raporty naprawcze i naprawiłem zidentyfikowane podatności CVE w kodzie.
\end{itemize}

\cvproject{Analiza Biometryczna Dynamiki Pisania}{Projekt Badawczy}{Python, Machine Learning, Scikit-learn}
\begin{itemize}[leftmargin=1.5em, label=\textbullet]
    \item \textbf{Uwierzytelnianie behawioralne}: Opracowałem model uczenia maszynowego do uwierzytelniania użytkowników na podstawie wzorców pisania (czas lotu, czas nacisku).
    \item \textbf{Detekcja anomalii}: Osiągnąłem wynik \textbf{AUC \textasciitilde0.81} w odróżnianiu użytkowników uprawnionych od intruzów przy użyciu klasyfikatorów Random Forest.
    \item \textbf{Zastosowanie}: Zademonstrowałem skuteczność biometrii behawioralnej jako warstwy ciągłego uwierzytelniania.
\end{itemize}

\cvproject{Symulacja Ataku i Obrony Sieciowej}{Home Lab}{Bettercap, Wireshark, Snort, ffuf}
\begin{itemize}[leftmargin=1.5em, label=\textbullet]
    \item \textbf{Analiza ruchu}: Symulowałem ataki MitM (ARP Spoofing, DNS Spoofing), aby zrozumieć wektory ataku i sygnatury detekcji.
    \item \textbf{Fuzzing aplikacji}: Przeprowadziłem rozległe testy fuzzingowe przy użyciu \textbf{ffuf}, odkrywając ukryte endpointy i pliki konfiguracyjne.
    \item \textbf{Detekcja}: Skonfigurowałem reguły IDS (Snort) do wykrywania i alarmowania o podejrzanych wzorcach sieciowych.
\end{itemize}

% Umiejętności
\cvsection{Umiejętności Techniczne}

\begin{itemize}[leftmargin=1.5em, label={}]
    \item \textbf{Security Operations}: SIEM (ELK Stack), Zarządzanie Podatnościami (OpenVAS, Nessus), IDS/IPS (Snort), Honeypots (T-Pot).
    \item \textbf{Web Security}: OWASP Top 10, Burp Suite, OWASP ZAP, Security Headers, koncepcje WAF, Bezpieczeństwo Uploadu Plików.
    \item \textbf{Sieci}: TCP/IP, Model OSI, Wireshark, Nmap, VPN, DNS, DHCP, Routing (podstawy BGP).
    \item \textbf{Chmura i IAM}: Microsoft Azure, Active Directory, Authentik, OIDC, SAML, Docker, Linux (Kali, Ubuntu).
    \item \textbf{Programowanie}: Python (Skrypty, Automatyzacja), SQL, Bash, JavaScript (Podstawy).
\end{itemize}

% Języki
\cvsection{Języki Obce}
\begin{itemize}[leftmargin=1.5em, label=\textbullet]
    \item \textbf{Angielski}: C1 (Zaawansowany) -- Pełna biegłość zawodowa
    \item \textbf{Polski}: Ojczysty
\end{itemize}

% Wykształcenie
\cvsection{Wykształcenie}

\educationentry{Magister Informatyki (Cyberbezpieczeństwo)}{Politechnika Rzeszowska}{2025 -- Obecnie}{Specjalizacja: Cyberbezpieczeństwo i Technologie Chmurowe}

\educationentry{Inżynier Elektroniki i Telekomunikacji}{Politechnika Rzeszowska}{2019 -- 2025}{Specjalizacja: Systemy Telekomunikacyjne}

% Certyfikaty
\cvsection{Certyfikaty i Szkolenia}
\begin{itemize}[leftmargin=1.5em, label=\textbullet]
    \item \textbf{CompTIA Security+} (SY0-701) -- \textit{W trakcie/Ukończone}
    \item \textbf{Google Cloud Digital Leader} -- Google Cloud Skills Boost
    \item \textbf{Ścieżki TryHackMe}: Pre-Security, Web Fundamentals, Jr. Penetration Tester, SOC Level 1
    \item \textbf{Bezpieczeństwo Sieci} -- Certyfikat ukończenia kursu
\end{itemize}

\end{document}